
\section{Overview}
In the field of computer vision, the techniques for synthesizing virtual view images from a number of real camera
images have been studied since the 1990s \cite{b1,b2,b3}.
% Free-viewpoint video in soccer TV broadcast
% production 
% The requirements for free-viewpoint video in sports TV broadcast
% production
Free-viewpoint video in sports TV broadcast production is a challenging problem involving the conflicting requirements of 
broadcast picture quality with video-rate generation.
FVV techniques for generating novel viewpoints using a multiview camera setup can be catogorized into two classes 
\cite{05_plane_sweeping}: 3D reconstruction and image-based rendering. 

Using 3D reconstruction, it is possible to construct 3D models of objects that provides a geometric proxy which can be
used to combine observations from multiple views in order to render images from an arbitrary viewpoint. 
Several approaches have been realized for reconstruction: visual-hull, photo-hull, stereo, and 
global shape optimisation \cite{02_iview}.
% TODO: si può scrivere di più da 02
The quality of the virtual view image generated
by these methods depends on the accuracy of the 3D model. In order to produce an accurate model, a
large number of video cameras surrounding the object should be used. 
Although 3D reconstruction is robust, artifacts such as ghosting objects can be introduced if a lot of ojects are in the scene. 

Image-based methods generate the image of the novel viewpoint directly without explicitly reconstructing the 3D scene structure.
The quality of rendered views depends on the accuracy of alignment between multiple view observations 
\cite{05_plane_sweeping,02_iview}.
Usually, these methods are limited to rendering viewpoints between the camera views.
% In many
% methods, depth generation, implicit or explicit, is
% done concurrently.



% Moreover, ... TODO:01
% Furthermore,
% camera calibration [13] is usually required to relate 2-D co-
% ordinates in images to their corresponding 3-D coordinates in
% object space. As it is essential to measure the 3-D positions of
% several points in the object space, calibration becomes difficult
% especially in a large space. For these reasons, the object area is
% generally limited to a few cubic meters in this approach.

% TODO: overview dei sistemi esistenti presente in 02