\section{Introduction}
Nowadays, sport broadcasting plays a large role in current society.
Therefore, it is important to provide a high quality and visually pleasing reporting of sports events.
As we know, incidents in sport tend to be over very quickly.
Slow-motion replays can be used to illustrate these incidents as clearly as possible for the viewer. 

Although time is stretched in these replays, there is no exploration of the spatial scene information, which is usually 
important for understanding the event.
A system that allows a replay from any angle adds a lot of value to the viewer experience.

% Especially sports broadcast-
% ing is a well-known recreational aspect of television.
% Therefore, it is important to provide a high quality and
% visually pleasing reporting of sports events.

% In sport most interesting incidents tend to be over very quickly.
% A system that allows a replay from any angle adds a lot of
% value to the production of sport coverage. Sports producers
% may use techniques such as slow-motion replays to illustrate
% these incidents as clearly as possible for the viewer. Although
% time is stretched in these replays, there is no exploration of
% the spatial scene information, which is usually important for
% understanding the event.

Free-viewpoint video (FVV) is one of the new trends in the development of advanced visual media type
that provides an immersive user experience and interactivity when viewing
visual media. Compared with traditional fixed-viewpoint
video, it allows users to select a viewpoint interactively and
is capable of rendering a new view from a novel viewpoint.
% Since a virtualized reality system \cite{b4} that distributes 51
% cameras over a 5 m done with controlled lighting and well-calibrated
% cameras was introduced, 
FVV has been a research topic in the field of computer vision 
since the virtualized reality system \cite{b4} was developed,
ranging from static models for studio applications with a fixed
capture volume, controlled illumination and backgrounds \cite{b5} 
to dynamic object models for sports scenes \cite{b6,b7,b8}.
Live outdoor sports such as soccer involve a number of additional challenges for both acquisition and processing phases. 
The action take place over an entire pitch and video acquisition should be done at sufficient resolution in order to
do analysis and production of desired virtual camera views.
% Moreover, this has not been confined to academia,
% the companies LiberoVision, Intel, and 4DViews also attach
% importance to the technique and have been providing visual
% effects applications for various purposes.

In this paper we briefly present and compare some methods to accomplish free-viewpoint video 
visualisation for soccer scenes... % TODO


% TODO: aggiungere da qualche parte
% There is a demand from the broadcast industry for more flexible production tech-
% nologies at live events such as sports. The ability to place virtual
% cameras at any location around or on the pitch is highly attractive to broadcasters
% greatly increasing flexibility in production and enabling novel delivery formats such
% as mobile TV. Examples of the type of physically impossible camera views which
% could be desirable are the goal keepers view, a player tracking camera or even a
% ball camera.